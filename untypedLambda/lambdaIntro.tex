\chapter{Einleitung}

Das Lambdakalkül ist eine (typenlose) Theorie die das Konzept Funktion als 
Abbildungsvorschriften und nicht, wie in der modernen Mathematik verbreiteter, 
als Graphen erfasst\autocite{Barendregt2012}.

\paragraph{Funktionen als Abbildungsvorschriften }

Bei der Formalisierung einer Funktionen als Abbildungsvorschrift wird 
diese als die explizite Anweisung kodiert, 
wie Eingabewerte -- die Argumente der Funktion -- 
auf Ausgabewerte abgebildet werden.
Diese Vorschrift ist die \emph{Definition} der Funktion.

Diese Darstellung ist intensional -- \todo{Was genau bedeutet intensional hier?}.

\todo{vgl. Algorithmus -- ist eine Funktion in diesem Sinn ein Algorithmus?}
\todo{Anmerkung: offensichtlich eng verwandt: Konzept der berechenbaren Funktion}
\todo{Anmerkung: Reifiziert: Input/Output-Charakter von Funktionen}
\todo{Anmerkung: Reifiziert: Computationale Aspekte der Funktion}
\todo{Recherchieren: Ist das die ältere Auffassung einer Funktion?
	  \url{https://en.wikipedia.org/wiki/History_of_the_function_concept}}

\paragraph{Funktionen als Graph}

Bei der Formalisierung einer Funktionen als Graph 
wird diese als spezielle Relation aufgefasst: 
als Menge von Paaren bei denen jedes erste Element in Relation zu genau
einem zweiten Element steht.

\begin{comment}
Definition Funktionsgraph: https://de.wikipedia.org/wiki/Funktionsgraph
\end{comment}

Diese Darstellung ist extensional -- \todo{Was genau bedeutet extensional hier?}.

\todo{
Recherchieren: Was genau ist der Vorteil von Funktionen dargestellt als Graphen.
Warum ist dieses Konzept besser zu handhaben und hat sich durchgesetzt?
Funktionen dargestellt als Abbildungsvorschriften sind nicht immer
automatisiert vergleichbar -- siehe Halteproblem.
}

\paragraph{Essentielle Eigenschaften des Konzepts 
			\enquote{Funktionen als Abbildungsvorschriften}}

Das Konzept \enquote{Funktionen als Abbildungsvorschriften} wird 
in seiner umfangreichsten Extension verstanden;
Funktionen können sowohl in (Fragmenten von) natürlichen Sprachen 
reprästentiert werden (Deutsch, Englisch, \dots),
sowie in formalen Sprachen (Programmiersprachen, \dots \todo{ergänzen}).

All diese Repräsentationsformen teilen gemeinsame essentielle Eigenschaften,
welche auch den Ausgangspunkt für die Theorie des Lambdakalküls bilden:
\begin{enumerate}
\item \label{typefree} Es sind typfreie Strukturen.
\item \label{fundata}  Objekte dieser Struktur können sowohl als Funktion 
	  sowie als Daten interpretiert werden.
\end{enumerate}

Aus \ref{typefree} und \ref{fundata} folgt eine wichtige Eigenschaft:
Funktionen in dieser Darstellungsweise können auf sich selbst angewendet werden.
Dies wird bei der Formalisierung von Funktionen als Graphen 
(damit auch in mengentheoretischen Darstellungen wie \emph{ZFC}) 
durch das \emph{Fundierungsaxiom} (auch \emph{Regularitätsaxiom})
verhindert.

\todo{
Recherchieren: Was genau sagt das Fundierungsaxiom aus?
\url{https://de.wikipedia.org/wiki/Zermelo-Fraenkel-Mengenlehre}
}

% ------------------------------------------------------------------------------
\newpage
\section{Aspekte des untypisierten Lambdakalküls}

Das untypisiere Lambdakalkül kann aus drei Perspektiven betrachtet werden:
\begin{enumerate}
\item Pures Lambdakalkül:\\
	Die erste Perspektive auf das Lambdakalkül fokussiert das eigentliche System:
	das formale Objekt wird hier ohne Rekursion auf einen intendierten 
	Verwendungszweck untersucht.
	
	Natürlich gelten Ergebnisse die in diesem Bereich erzielt werden auch in den
	Anwendungsbereichen des Kalküls.
\item Lambdakalkül in der Informatik:\\
	Die zweite Perspektive ist die Sicht der Informatik:
	Das Lambdakalkül hat in dieser Wissenschaft eine große und, 
	mit der zunehmenden Verwendung funktionaler Programmiersprachen, 
	weiter wachsende Bedeutung. 
\item Lambdakalkül als Grundlage der Mathematik:\\
	Die dritte Perspektive untersucht das Lambdakalkül hinsichtlich
	seiner Eigung als unifizierende Darstellung von Logik und Mathematik.
\end{enumerate}

\todo{erwähnen: Lambdakalkül in der Linguistik}

% ------------------------------------------------------------------------------
% ------------------------------------------------------------------------------
\newpage
\section{Pures Lambdakalkül}

In diesem Abschnitt soll das Lambdakalkül als solches eingeführt,
und ohne Bezugnahme auf seine weitere mögliche Verwendung untersucht
werden. Selbstverständlich gelten alle in diesem Abschnitt vorgestellten
Ergrebnisse auch unter einer Interpretation des Systems als Programmiersprache
und alle weitere Verwendungen.


% ------------------------------------------------------------------------------
\subsection{Definition}

\begin{comment}
weitere untergliederung:
Definition
Conversion
Reduction
Theorien
Modelle
\end{comment}

\paragraph{Syntaktische Primitive des Lambdakalküls: Terme der Theorie}
% wörtliche übersetzung

Zentraler Gegenstand des Lambdakalküls sind Funktionen.
Zwei zentrale Eigenschaften bilden den Ausgangspunkt der Theorie:
\begin{enumerate}
\item \label{funmap} Die explizite Kodierung von Funktionen und 
	Funktionsargumenten als deklarative Abbildungsvorschriften.
\item \label{funio} 
	Die Eigenschaften dieser Funktionsterme bei Anwendungen auf Argumente.
\end{enumerate}

Terme des Lambdakalküls: % Terme denotieren Funktionen
Die Terme der Theorie $λ$ besteht aus der Menge $Λ$ der Lambdaterme.
Diese Terme werden induktive aus den drei primitiven der Sprache aufgebaut:
Abstraktion, Applikation und Variablen.

\begin{definition}[Syntax des Lambdakalküls]
Aus \ref{funmap} und \ref{funio} ergeben sich die 
drei primitiven Sprachelemente des Kalküls:
\begin{verbatim}
Λ = Var		(Variable)
  | λ.Λ		(Abstraktion)
  | Λ Λ		(Applikation)
\end{verbatim}
\end{definition}

Abstraktion und Applikation sind Gegenoperationen.

\paragraph{Semantische Primitive des Lambdakalüls: Formeln der Theorie}

Formeln/Aussagen des Lambdakalküls: % Formeln denotieren Wahrheitswerte
Die innerhalb des Systems formulierten Aussagen sind Gleichungen und 
damit Identitätsaussagen zwischen Lambdatermen.
\todo{Zusammenhang: Equational Reasoning?}

\begin{definition}[Axiom β (Betareduktion)]

Das Kalkül hat als einzige Axiome Instanzen des Axiomenschemas $β$.

\begin{verbatim}

λx.Λ Λ = Λ Λ 

\end{verbatim}

\end{definition}



\paragraph{Unäre Funktionen und Currying}
% wörtliche übersetzung
Im Lambdakalkül werden ausschließlich unäre Funktionen behandelt;
allerdings stellt dies keine Einschränkung dar:
n-äre Funktionen können durch eine \enquote{Schönfinkeln/Currying}
genannte Methode auf bedeutungsgleiche unäre funktionen abgebildet werden. 

\todo{
Ist wohl ein sehr wichtiges Prinzip:
Semantische Konsequenzen, iwas mit Kartesisch geschlossenen Kategorien S.6
}



\paragraph{Intensionaler Charakter des Systems}
% wörtliche übersetzung
Das Konzept von Funktionen als Abbildungsvorschriften % Algorithmen
ist intensional.
Aufgrund dieses intensionalen Charakters sind die linguistischen Mittel 
diese Funktionen zu beschreiben -- die Lambdaterme -- von großer Bedeutung.


\paragraph{Verbindung des Lambdakalküls und der Kategorientheorie}
% wörtliche übersetzung
Gegenstand beider Theorien sind Funktionen, allerdings aus unterschiedlichen
Blickwinkeln mit verschiedenen Perspektiven:
\begin{itemize}
\item Das Lambdakalkül ist eine intensionale Formalisierung 
	  von Funktionen als Algorithmen und deren Anwendungen auf ebenfalls
	  als Funktionen definierte Argumente
\item Die Kategorientheorie ist eine algebraische Formalisierung
	  von Funktionen unter Komposition \todo{ergänzen}
\end{itemize}



% ------------------------------------------------------------------------------
\subsection{Konversion}



% ------------------------------------------------------------------------------
\subsection{ReduKtion}



% ------------------------------------------------------------------------------
\subsection{Theorien}



% ------------------------------------------------------------------------------
\subsection{Modelle}



% ------------------------------------------------------------------------------
% ------------------------------------------------------------------------------
\newpage
\section{Lambdakalkül in der Informatik}



% ------------------------------------------------------------------------------
\subsection{Lambdakalkül in der Berechenbarkeitstheorie}

Zum einen ist das Kalkül in höchstem Maß relevant für die Grundlagen
der Berechenbarkeitstheorie und Rekursionstheorie.
In den 1930er Jahren wurde das bislang nur informal verwendete Konzept 
der \enquote{effektiv berechenbaren Funktion} auf drei verschiedene Arten
formalisiert:

\begin{enumerate}
\item $μ$-rekursive Funktion bzw. partiell rekursive Funktionen.
	  \todo{Fußnote: von Gödel und Jaques Herbrand)}
\item Turing-berechenbare Funktionen basierend 
	  auf einem abstrakten Maschinenmodell: den Turingmaschinen.
	  \todo{Fußnote: von Alan Turing}
\item $λ$-definierbare Funktionen.
	  \todo{Fußnote: von Alonzo Chruch}
\end{enumerate}

Obwohl jede diese Formalisierungen einen anderen Aspekt berechenbarer Funktionen
als Ausgangspunkt, und damit auch andere primitiven als Sprachelemente reifiziert,
konnte durch Turing und Church gezeigt werden, dass alle drei Konzepte den 
gleichen Gegenstandsbereich erfassen. 
\todo{Gegenstandsbereich erfassen ... ist das korrekt?}

Folgt man der Churchschen These ist dies die 
Klasse der effektiv berechenbaren Funktionen:

\begin{definition}{Churchsche These}
Die durch die formale Definition der Turing-Berechenbarkeit
(Äquivalent: While-Berechenbarkeit, Goto-Berechenbarkeit,
$μ$-Rekursivität oder $λ$-Definierbarkeit) erfasste Klasse von Funktionen
stimmt genau mit der Klasse der im intuitiven Sinn berechenbaren 
Funktionen überein.
\end{definition}


% ------------------------------------------------------------------------------
\subsection{Lambdakalkül als paradigmatische Progammiersprache}

Zum anderen besitzt das Lambdakalkül die wesentlichen Elemente
einer Programmiersprache:
Trotz seiner Einfachheit % \begin{comment} wie in Teil ... dargestellt \end{comment}
ist es gemäß der Churchschen These stark genug um alle intuitiv berechenbaren
Funktionen auszudrücken;
gerade wegen seiner schlichten Eleganz kann es als die prototypische
Programmiersprache aufgefasst werden die ein ganzes Paradigma begründet --
das der funktionalen Programmiersprachen.

Der Grad in dem verschiedene Programmiersprachen anleihen vom Lambdakalkül 
nehmen ist dabei fließend: 
Das Spektrum reicht von der Übernahme einiger syntaktischer Konstrukte 
wie anonymen Funktionen bishin zu einer größtmöglichen semantischen 
Korrespondenz.

Moderne funktionale Programmiersprachen beziehen neben syntaktischen Anleihen
sowohl ihre operationale als auch denotationale Semantik in großer 
Übereinstimmung mit den dazu korrespondierenden Semantiken des Lambdakalküls.
\todo{vll. besser: ein vorteil denotatineller Semantiken im Vergleich
zu operationalen Semantiken ist ...}
Ein wesentlicher Vorteil dieses Ansatzes im Vergleich zu den informellen 
Semantiken prozeduraler Programmiersprachen ist die 
vergleichsweise einfache funktionale Interpretation von Programmen 
dieser Sprachen -- es handelt sich nicht um Zustandstransformatoren
die zwischen oftmals nur implizit gegebenen Maschinenzuständen abbilden,
sondern um Algorithmen in deklarativer Darstellung die von dem
zu Grunde liegenden Maschinenmodell und dessen Konfigurationen abstrahieren.
Damit sind alle (oder zumindest ein großer Teil der) Wahrheiten gültig 
die über das Lambdakalkül ausgesagt werden können.
Dies erleichtert die Analyse komplexer Systeme und erlaubt die unkomplizierte
Anwendung formaler mathematischer Methoden bei der Analyse von Programmen.

\todo{
Anmerkung zur zunehmenden Verwendung funktionaler Elemente in objektorientierten
Programmiersprachen
}

\todo{
Verweis auf die Verschiedenen Arten von Semantiken (operational, denotationell,
...
}

\begin{comment}
Dieser Teil behandelt eine Übersicht über die denotationelle Semantik 
des Lambdakalküls ... vll. besser an eine andere Stelle 
\end{comment}

\paragraph{Denotationelle Semantik und das untypisierte Lambdakalkül}
% direkt übersetzt nach Barendregt
Die Objekte des untypisierten Lambdakalküls sind -- 
auf Grund der Typfreiheit des Systems -- nicht stratifiziert.
Es ist daher nicht trivial ein Modell für das System zu finden: %besser konstruieren?
eine Menge $X$ in welche ihr Funktionenraum $X → X$ eingebettet ist.
Im allgemeinen Fall ist dies auf Grund der zunehmenden Kardianlität unmöglich.

Scott konnte dieses Problem 1969 lösen:
Modelle für das Lambdakalkül können konsturiert werden in dem 
der Funktionenraum $X → X$ auf kontinuierliche \todo{continuous Definition?}
Funktionen über $X$ eingeschränkt wird.

Dieser Ansatz ist leistungsstark genug um den folgenden wichtigen
Bestandteilen funktionaler Programmiersprachen eine Bedeutung zu geben:
Rekursion (least fixpoint) und (algebraischen) Datentypen.

Scott-Semantiken sind allerdings in einem wichtigen Punkt schwach:
sie erfassen bis auf wenige Ausnahmen nur sequentielle Funktionen.
\todo{recherchieren ... Was ist mit referentieller Transparenz?
wie werden hier sequentielle Funktionen definiert?}
Einige parallele Funktionen wie \enquote{parallel or} können 
auch mit dieser Semantik modelliert werden.



% ------------------------------------------------------------------------------
% ------------------------------------------------------------------------------
\newpage
\section{Lambdakalkül als Grundlage der Mathematik}

Die Schöpfer des Lambdakalküls \todo{Wer genau?} wollten mittels 
diesen Systems zwei Ziele erreichen:
\begin{enumerate}
\item \label{funtheory} Eine generelle Funktionentheorie zu entwickeln
\item \label{exfuntheory} Diese Funtkionentheorie um logische Konzepte 
	  zu ergänzen und damit ein System zu erhalten, 
	  in dem Logiken \todo{Logiken oder eine Logik?} 
	  und Teile der Mathematik einheitlich formuliert werden können.
	  \todo{Was muss eine Grundlage der Mathematik überhaupt leisten können?}
\end{enumerate}

Die Entwicklung einer konsistenten Funktionentheorie wurde erfolgreich
von Curry vorangetrieben: Resultat ist die Kombinatorenlogik. 
\todo{wird die noch vorgestellt?}

Bislang ist es noch nicht gelungen das Lambdakalkül als grundlegendes
System der Mathematik zu verwerden:
Churchs ursprüngliches System war inkonsistent, wie durch das 
Kleene-Rosser-Paradoxon bewiesen wurde. \todo{Jahreszahl}
Weitere Ansätze, die zum Teil auf dem unvollendeten Werk
von Curry beruhen, werden noch verfolgt.

Ansätze die an Stelle des untypisierten Lambdakalküls typisierte Varianten
verwenden sind bislang wesentlich erfolgreicher -- 
unter Ausnutzung der Curry-Howard-Korrespondenz, \todo{...Lambek?}
die eine Verbindung zwischen typisierten Lambdakalkuli, verschiedenen Logiken
und Kategorientheorie herstellt, 
konnten weitreichende Ergebnisse erzielt werden.

Die Verbidungen dieser drei Bereiche wird im weiteren Verlauf dieser Arbeit
eine bedeutende Rolle spielen.


\todo{
Anmerkung: zentrales Thema bei Funktionen und ... - 
finitisierung des Transfinisierung des Transfiniten
}



